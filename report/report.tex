\documentclass[12pt,a4paper]{article}
\usepackage[UTF8]{ctex}
\usepackage{amsmath}
\usepackage{amsfonts}
\usepackage{amssymb}
\usepackage{graphicx}
\usepackage{float}
\usepackage{hyperref}
\usepackage{geometry}
\usepackage{listings}
\usepackage{color}

\geometry{left=2.5cm,right=2.5cm,top=2.5cm,bottom=2.5cm}

\definecolor{dkgreen}{rgb}{0,0.6,0}
\definecolor{gray}{rgb}{0.5,0.5,0.5}
\definecolor{mauve}{rgb}{0.58,0,0.82}

\lstset{frame=tb,
  language=Matlab,
  aboveskip=3mm,
  belowskip=3mm,
  showstringspaces=false,
  columns=flexible,
  basicstyle={\small\ttfamily},
  numbers=none,
  numberstyle=\tiny\color{gray},
  keywordstyle=\color{blue},
  commentstyle=\color{dkgreen},
  stringstyle=\color{mauve},
  breaklines=true,
  breakatwhitespace=true,
  tabsize=3
}

\title{Assignment 2 报告}
\author{朱林-2200011028}
\date{\today}

\begin{document}

\maketitle

\section{引言}
计算固体力学第二次作业报告,包含以下问题的有限元方法 (FEM) 实现过程:
\begin{enumerate}
    \item \textbf{Problem 1}: 二维线性弹性力学 (平面应力)。
    \item \textbf{Problem 2}: Mindlin 板弯曲问题。
\end{enumerate}

\section{Problem 1: 二维线性弹性分析}

\subsection{问题描述}
本部分代码旨在求解二维平面应力问题。在默认配置下,代码模拟了以下物理问题:
\begin{itemize}
    \item \textbf{几何模型}: $1\text{m} \times 1\text{m}$ 的正方形板。
    \item \textbf{边界条件}: 左侧边缘 ($x=0$) 完全固定 ($u_x=0, u_y=0$)。
    \item \textbf{载荷条件}: 顶部边缘 ($y=1\text{m}$) 施加均匀分布的向下面力 (Traction),大小为 $T_y = -2 \times 10^4 \text{ N/m}$。
    \item \textbf{材料属性}: 杨氏模量 $E = 2.1 \times 10^9 \text{ Pa}$,泊松比 $\nu = 0.3$。
\end{itemize}
该模型类似于一个左端固定的悬臂板,在顶部受到剪切/压缩载荷。

\subsection{实现细节}

\subsubsection{网格生成 (\texttt{generate\_mesh.m})}
网格生成函数创建了结构化的节点和单元网格。
\begin{itemize}
    \item \textbf{节点}: 根据划分数量 ($n_x, n_y$) 遍历 $x$ 和 $y$ 坐标生成。
    \item \textbf{单元}:
    \begin{itemize}
        \item \textbf{Tri3 (Flag=1)}: 每个矩形网格单元被划分为两个三角形。
        \item \textbf{Quad4 (Flag=2)}: 每个矩形网格单元形成一个四边形单元。
    \end{itemize}
\end{itemize}

\subsubsection{刚度矩阵组装 (\texttt{K\_matrix.m})}
通过遍历所有单元组装全局刚度矩阵 $K$:
\begin{equation}
K = \sum_{e=1}^{M} K_e = \sum_{e=1}^{M} \int_{\Omega_e} B^T D B \, d\Omega
\end{equation}
对于 Quad4 单元,使用 $2\times2$ 高斯积分进行数值积分。对于 Tri3 单元,1点积分(形心)对于常应变三角形 (CST) 已经足够。

\subsubsection{边界条件 (\texttt{Boundary\_conditions.m})}
边界条件定义如下:
\begin{itemize}
    \item \textbf{位移边界}: 左边缘 ($x=0$) 完全固定 ($u_x=0, u_y=0$)。这是通过识别 $x$ 坐标接近 0 的节点来实现的。
    \item \textbf{力边界}: 在顶边缘 ($y=H$) 施加面力载荷。\texttt{l\_area} 向量存储边界上每个节点的有效长度/面积,用于将力转换为节点力。
\end{itemize}

\subsection{结果}
计算了位移场和应力场。由于不习惯原 plot 函数,对于原始文件中的绘图代码进行了调整以生成更清晰的云图。
\begin{figure}[H]
    \centering
    \includegraphics[width=0.8\textwidth]{../result/problem1/displacement_contour.png}
    \caption{Problem 1 位移云图}
\end{figure}
\begin{figure}[H]
    \centering
    \includegraphics[width=0.8\textwidth]{../result/problem1/stress_contour.png}
    \caption{Problem 1 应力云图}
\end{figure}

\section{Problem 2: Mindlin 板弯曲分析}

\subsection{问题描述}
分析一个在中心受高斯分布载荷作用的方形板 ($1\text{m} \times 1\text{m}$,厚度 $t=0.05\text{m}$)。四边均简支($w=0$)。材料属性为 $E=210$ GPa 和 $\nu=0.3$。

\subsection{理论公式}
问题采用 Mindlin-Reissner 板理论建模,该理论考虑了剪切变形。每个节点有 3 个自由度 (DOFs):
\begin{itemize}
    \item $w$: 横向挠度。
    \item $\theta_x$: 绕 x 轴的转角。
    \item $\theta_y$: 绕 y 轴的转角。
\end{itemize}

单元刚度矩阵由弯曲和剪切两部分组成:
\begin{equation}
K_e = K_b + K_s = \int_{\Omega_e} B_b^T D_b B_b \, dA + \int_{\Omega_e} B_s^T D_s B_s \, dA
\end{equation}

\subsection{实现细节}

\subsubsection{剪切自锁控制}
为了防止薄板中的剪切自锁(特别是对于 Quad4 单元),对剪切刚度矩阵 $K_s$ 采用了 减缩积分 (Reduced Integration):
\begin{itemize}
    \item \textbf{弯曲刚度 ($K_b$)}: 使用全积分计算(Quad4 使用 $2\times2$ 高斯点)。
    \item \textbf{剪切刚度 ($K_s$)}: 使用减缩积分计算(Quad4 使用 $1\times1$ 高斯点)。
\end{itemize}
确保了单元在薄板极限下不会变得过硬。

\subsubsection{高斯载荷积分 (\texttt{F\_vector.m})}
通过使用形函数在单元面积上积分高斯压力场 $q(x,y)$ 来计算外力向量:
\begin{equation}
F_e = \int_{\Omega_e} N^T q(x,y) \, dA
\end{equation}

\subsection{不同网格密度下的计算}
使用三种网格密度进行了计算:
\begin{itemize}
    \item 粗网格: $5 \times 5$ 单元。
    \item 中等网格: $10 \times 10$ 单元。
    \item 细网格: $50 \times 50$ 单元。
\end{itemize}
测试了 Tri3 和 Quad4 两种单元。结果表明,随着网格密度的增加,中心挠度收敛于一个稳定值。在相同节点数下,Quad4 单元通常比 Tri3 单元表现出更快的收敛速度。

\subsection{结果与验证}
\subsubsection{MATLAB 仿真结果}
模拟输出包括:
\begin{itemize}
    \item \textbf{挠度 ($w$)}: 最大挠度出现在板中心。
    \item \textbf{转角幅值}: 中心处为零(由于对称性),边缘附近最大。
    \item \textbf{Von Mises 应力}: 基于弯矩在板顶表面 ($z=t/2$) 计算得出。
\end{itemize}

\begin{figure}[H]
    \centering
    \includegraphics[width=0.8\textwidth]{../result/problem2/Quad4_50x50/1_Deflection_w.png}
    \caption{挠度云图 (Quad4, 50x50 网格)}
\end{figure}
\begin{figure}[H]
    \centering
    \includegraphics[width=0.8\textwidth]{../result/problem2/Quad4_50x50/3_Stress_VonMises.png}
    \caption{Von Mises 应力云图 (Quad4, 50x50 网格)}
\end{figure}

\subsubsection{与商业软件 (Abaqus) 的对比验证}
为了验证 MATLAB 代码的准确性,使用商业有限元软件 Abaqus 进行了对比分析。
\begin{itemize}
    \item \textbf{单元类型}: S4R (4节点减缩积分壳单元),与 MATLAB 中的 Quad4 Mindlin 板单元理论一致。
    \item \textbf{网格密度}: $50 \times 50$ (2500 个单元)。
    \item \textbf{积分方案}: 厚度方向采用 Simpson 积分 (5个积分点)。
    \item \textbf{载荷施加}: 使用 Analytical Field 定义高斯分布载荷。
\end{itemize}

表 \ref{tab:comparison} 展示了 MATLAB 代码与 Abaqus 仿真结果的对比。选取了两个关键指标:板中心的最大挠度 ($w_{max}$) 和板中心的最大 Von Mises 应力 ($\sigma_{vm, max}$)。

\begin{table}[H]
    \centering
    \caption{MATLAB 代码与 Abaqus 结果对比 (50x50 网格)}
    \label{tab:comparison}
    \begin{tabular}{|c|c|c|c|}
        \hline
        \textbf{指标} & \textbf{MATLAB (Quad4)} & \textbf{Abaqus (S4R)} \\
        \hline
        中心挠度 $w_{max}$ (m) & -1.088290e-03 &  -1.08737e-03  \\
        \hline
        最大 Von Mises 应力 (Pa) &   1.026135e+08  & 1.02514e+08 \\
        \hline
    \end{tabular}
\end{table}

图 \ref{fig:3d_comparison} 展示了 MATLAB 代码计算得到的 3D 变形图与 Abaqus 仿真结果的对比。两者在变形趋势和幅值分布上高度一致。

\begin{figure}[H]
    \centering
    \begin{minipage}[b]{0.48\textwidth}
        \centering
        \includegraphics[width=\textwidth]{../result/problem2/Quad4_50x50/5_Deformed_Shape_3D.png}
        \caption{MATLAB 3D 变形图}
    \end{minipage}
    \hfill
    \begin{minipage}[b]{0.48\textwidth}
        \centering
        \includegraphics[width=\textwidth]{fig/abaqus_3d_result.png}
        \caption{Abaqus 3D 变形图}
    \end{minipage}
    \caption{MATLAB 与 Abaqus 3D 变形结果对比}
    \label{fig:3d_comparison}
\end{figure}

计算结果与 abaqus 存在一定差异,可能原因包括:
\begin{enumerate}
    \item \textbf{单元类型差异}: 虽然 S4R 单元与 Quad4 Mindlin 板单元类似,但在数值实现细节上可能存在差异。
    \item \textbf{积分方案}: Abaqus 使用了厚度方向的 Simpson 积分,而 MATLAB 代码中未明确考虑厚度积分,可能导致结果偏差。
    \item \textbf{网格划分}: 尽管网格数量相同,但节点分布和单元形状可能略有不同,影响结果精度。
    \item \textbf{边界条件实现}: 边界条件的具体实现方式可能存在差异,影响整体响应。
\end{enumerate}


\end{document}
